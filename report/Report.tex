\documentclass[a4paper]{article}

\usepackage{pgf}
\usepackage[utf8]{inputenc}
\usepackage{verbatim}
\usepackage{titling}
\usepackage{booktabs}
\usepackage{enumitem}
\usepackage{qtree}
\usepackage{amssymb}
\usepackage{amsmath}
\usepackage{times}
\usepackage{dsfont}
\usepackage{titling}
\usepackage[a4paper,
bindingoffset=0.2in,
left=1in,
right=1in,
top=2in,
bottom=1in,
footskip=.25in]{geometry}


\pretitle{\begin{center}\linespread{1}}%\LARGE\bfseries \bigskip \rule{\textwidth}{2pt} \noindent
  \posttitle{\end{center}\vspace{0.14cm}} % \bigskip \noindent\rule{\textwidth}{2pt}
\preauthor{\begin{center}\Large}
  \postauthor{\end{center}}

\setlength{\droptitle}{-10em}
\title { \Large{Seminario de Ciencias de la Computaci\'on B}\protect\\
  \large{Heurísticas de Optimización Combinatoria}\protect\\
  \large{Problema del Ajente Viajero\\con Recocido Simulado}}


\date{\normalsize{Viernes, 17 de Marzo, 2023.}}
\author{\normalsize{Profesor: Canek Peláez Valdés}\protect\\
  \normalsize{Autor: Xin Wen Zhang Liu}}\vspace{0.2cm}


% \vspace{0.6cm}
% \normalsize{{Fecha de entrega: Mi\'ercoles 18 de Marzo de 2023}} \protect\\ \vspace{0.62cm}
% \normalsize {}
\clearpage




\begin{document}
\allowdisplaybreaks
\maketitle

\subsection*{El problema del agente viajero}
El problema del agente viajero es uno de los problemas de Optimización combinatoria m\'as estudiados . Una pregunta f\'acil de hacer pero dif\'icil de repsonder:
\begin{center}
  Dado un conjunto de ciudades y sus coordenadas en el plano cartesiano, ?`cu\'al es ek camino m\'as corto que visite cada ciudad exactamente una vez?
\end{center}
La manera m\'a directa de resolver este problema ser\'ia listar todas las prosibles combinaciones de caminos que pasen por las ciudades deseadas y comparar sus costos. Sin embargo el n\'umero de combinaciones diferentes con $n$ ciudades crece a la par de  $n!$, lo que hace que la cantidad de posibles combinaciones crezca a un paso colosal cuando incrementamos $n$.\\

Imaginemos que tenemos $10$ ciudades, entonces $10! = 10 \times 9 \dots \times 1 = 3,628,800$



\subsection*{Recocido Simulado}
El t\'ermino recocido (annealing) proveniente de la t\'ecnica 
\section*{Implementaci\'on}
\subsection*{Lenguaje de rpogramaci\'on}
\subsection*{El algoritmo}

\section*{Experimentaci\'on y resultados}

\section*{Conclusiones}




\end{document}